\documentclass[12pt]{article}
\usepackage{amsmath, amssymb, amsthm, xcolor}
\usepackage[most]{tcolorbox}
\usepackage{lmodern}
\usepackage{titlesec}

% Define custom colors
\definecolor{problemcol}{HTML}{FFFAE3}
\definecolor{solutioncol}{HTML}{E8F8F5}
\definecolor{stepcol}{HTML}{F6DDCC}
\definecolor{finalcol}{HTML}{EAF2F8}
\definecolor{darkborder}{HTML}{2C3E50}

% Styling for titles
\titleformat{\section}{\large\bfseries\sffamily\color{darkborder}}{}{0em}{}
\titleformat{\subsection}{\bfseries\sffamily\color{darkborder}}{}{0em}{}

% General tcolorbox styling
\tcbset{
    enhanced,
    breakable,
    boxrule=0.8pt,
    arc=3mm,
    auto outer arc,
    colframe=darkborder,
    coltitle=black,
    fonttitle=\bfseries\sffamily,
    drop shadow
}

\setlength{\parskip}{1em}
\setlength{\parindent}{0pt}

\begin{document}

% Problem Box
\begin{tcolorbox}[colback=problemcol, title=\textbf{Problem}]
Let $\Omega$ be an open subset of $\mathbb{C}$. Suppose $f$ is a meromorphic function on $\Omega$ having exactly one zero at $z_0$ and exactly one pole at $z_1$. Further, $m$ is the order of zero at $z_0$ and $n$ is the order of the pole at $z_1$. If $g : \Omega \to \mathbb{C}$ is analytic and $D$ is an open disc with center $a$ and radius $r$ such that $z_0, z_1 \in D$ and $D \subset \Omega$, then show that
\[
\frac{1}{2\pi i} \int_\gamma \frac{f'(z)}{f(z)} g(z)\, dz = m g(z_0) - n g(z_1),
\]
where $\gamma(t) = a + r e^{2\pi i t}$ for $0 \le t \le 1$.
\end{tcolorbox}

\vspace{1em}

% Solution Box
\begin{tcolorbox}[colback=solutioncol, title=\textbf{Solution}]

We apply the residue theorem to the meromorphic function $\displaystyle \frac{f'}{f}g$.

% Step 1
\begin{tcolorbox}[colback=stepcol, title=\textbf{Step 1: Behavior of \boldmath$\dfrac{f'}{f}$ at zeros and poles}, breakable]
Suppose $f$ has a zero of order $m$ at $z_0$. Then in a neighborhood of $z_0$, we can write
\[
f(z) = (z-z_0)^m h(z),
\]
where $h(z)$ is analytic and $h(z_0) \neq 0$. Differentiating,
\[
f'(z) = m(z-z_0)^{m-1}h(z) + (z-z_0)^m h'(z),
\]
so
\[
\frac{f'(z)}{f(z)} = \frac{m}{z-z_0} + \frac{h'(z)}{h(z)}.
\]
Hence, $\dfrac{f'}{f}$ has a simple pole at $z_0$ with residue
\[
\operatorname{Res}_{z=z_0} \frac{f'(z)}{f(z)} = m.
\]
\end{tcolorbox}

% Step 2
\begin{tcolorbox}[colback=stepcol, title=\textbf{Step 2: Behavior at poles}, breakable]
Suppose $f$ has a pole of order $n$ at $z_1$. Then in a neighborhood of $z_1$, we can write
\[
f(z) = \frac{1}{(z-z_1)^n}k(z),
\]
where $k(z)$ is analytic and $k(z_1) \neq 0$. Differentiating,
\[
f'(z) = -\frac{n}{(z-z_1)^{n+1}}k(z) + \frac{1}{(z-z_1)^n}k'(z),
\]
so
\[
\frac{f'(z)}{f(z)} = \frac{-n}{z-z_1} + \frac{k'(z)}{k(z)}.
\]
Hence, $\dfrac{f'}{f}$ has a simple pole at $z_1$ with residue
\[
\operatorname{Res}_{z=z_1} \frac{f'(z)}{f(z)} = -n.
\]
\end{tcolorbox}

% Step 3
\begin{tcolorbox}[colback=stepcol, title=\textbf{Step 3: Apply residue theorem}, breakable]
The function $\dfrac{f'}{f}g$ is meromorphic on $D$, and its only singularities inside $\gamma$ are at $z_0$ and $z_1$. By the Cauchy residue theorem,
\[
\frac{1}{2\pi i} \int_\gamma \frac{f'(z)}{f(z)}g(z)\, dz
= \sum_{z \in \{z_0, z_1\}} \operatorname{Res}_{z}\left[\frac{f'(z)}{f(z)}g(z)\right].
\]

At $z_0$:
\[
\operatorname{Res}_{z=z_0}\left[\frac{f'(z)}{f(z)}g(z)\right]
= \lim_{z\to z_0} (z-z_0)\frac{f'(z)}{f(z)}g(z)
= m g(z_0).
\]

At $z_1$:
\[
\operatorname{Res}_{z=z_1}\left[\frac{f'(z)}{f(z)}g(z)\right]
= \lim_{z\to z_1} (z-z_1)\frac{f'(z)}{f(z)}g(z)
= -n g(z_1).
\]
\end{tcolorbox}

% Final Answer
\begin{tcolorbox}[colback=finalcol, colframe=black, title=\textbf{Final Answer}]
\[
\boxed{
\frac{1}{2\pi i} \int_\gamma \frac{f'(z)}{f(z)}g(z)\, dz = m g(z_0) - n g(z_1)
}
\]
\end{tcolorbox}

\end{tcolorbox}

\end{document}
